
\subsection{MNIST Workflow}\label{mnist-workflow}

MNIST is a well-known program to detect handwritten digits. It
provides value for our work because it is well understood and is used
in many educational efforts. Also, we created a workflow that
integrates the UVA Rivanna HPC (the workflow can easily be adapted to other
machines). The nice feature about this application is that it can be
configured to run very quickly while still using various GPUs and
benchmark their runtimes for running several MNIST Python
programs. These programs include machine learning processing,
convolutional neural network, long short-term memory, recurrent neural
network, and others. The programs can be found on
GitHub~\cite{www-mnist-programs}.

As for the workflow, we adapted it not only to run one algorithm but
multiple in an iteration across the GPUs (similar to
\ref{fig:graph-view}).

On a successful run, the output will be receiving runtimes similar to:

\begin{table}[!ht]
\caption{MNIST Performance as obtained by cloudmesh-cc on various graphics cards using workflow scheduling}
    \centering
    \begin{tabular}{lr}
    \hline
        Name & Time \\ \hline
        a100 & 106.046 \\ 
        v100 & 138.087 \\ 
        rtx2080 & 138.048 \\
        k80 & 171.057 \\ 
        p100 & 202.055 \\
    \end{tabular}
    \label{table:mnist-times}
  \end{table}

\TODO{need pdf for workflow, but maybe embedded in gui}