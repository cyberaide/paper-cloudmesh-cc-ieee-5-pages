
\begin{figure}[htb]

{\scriptsize\begin{verbatim}
  $ cms vm start --cloud aws
  $ cms vm start --cloud azure
\end{verbatim}}

\caption{Simple VM management for hybrid clouds}
\label{fig:cms}
\end{figure}  


\subsection{Run the workflow}\label{run-the-workflow}

The workflow can be run easily via the GUI. We added a special set of
buttons to the workflow table and graph display to simplify running of
the workflow. Certainly, the workflow can also be activated while
calling the appropriate REST call, either through Python, the OpenAPI
docs page, or, for example, a curl call.
Figures~\ref{fig:workflow-run-a}-\ref{fig:workflow-run-c}
showcase various methods to run the example workflow, which is called
{\em workflow-example}.


\begin{figure}[htb]

{\scriptsize\begin{verbatim}
  $ curl -X 'GET' 'http://127.0.0.1:8000/
                   workflow/run/workflow-example?show=True'
         -H 'accept: application/json'
\end{verbatim}}% $

\caption{Running the example workflow with curl.}
\label{fig:workflow-run-a}
\bigskip

{\scriptsize\begin{verbatim}
  from cloudmesh.cc.workflowrest import RESTWorkflow
  rest = RESTWorkflow()
  result = rest.run_workflow('workflow-example')
\end{verbatim}}

\caption{Running the example workflow with cloudmesh RESTWorkflow API.}
\label{fig:workflow-run-b}
\bigskip

{\scriptsize\begin{verbatim}
  import requests
  url = 'http://127.0.0.1:8000/workflow/run/'\
        'workflow-example?show=True'
  r = requests.get(url)
  print(r)
\end{verbatim}}

\caption{Running the example workflow with requests API.}
\label{fig:workflow-run-c}

\end{figure}